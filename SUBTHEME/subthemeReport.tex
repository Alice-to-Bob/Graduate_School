% !TeX root = ./subthemeReport.tex

\documentclass[11pt,a4j,titlepage]{jreport}
\usepackage{indentfirst}
\usepackage[dvipdfmx,hiresbb]{graphicx}
\usepackage{comment}
\usepackage{here}
\usepackage{enumerate}
\usepackage{afterpage}

\usepackage{times}

\usepackage{url}
\usepackage{listings,jvlisting}

\renewcommand{\bibname}{参考文献}
\lstset{
    basicstyle={\ttfamily\small}, %書体の指定
    frame=tRBl, %フレームの指定
    framesep=10pt, %フレームと中身(コード)の間隔
    breaklines=true, %行が長くなった場合の改行
    linewidth=12cm, %フレームの横幅
    lineskip=-0.5ex, %行間の調整
    tabsize=2 %Tabを何文字幅にするかの指定
}

%\input{link.bib}
%各種パッケージ
%\usepackage{...}


\title{副テーマ研究報告書 \\ 隔離ネットワークにおける通信制御の調査}


\date{\today}\author{\\ \\ \\ 1910225 嶂南 秀敏\\ \\
\\ \vspace{1mm}副テーマ指導教員 知念 賢一 特任准教授
\\ \vspace{1mm}主指導教員 藤崎 英一郎 教授}


\begin{document}

\maketitle



\begin{abstract}
    
    
    現在、ユーザの遠隔サーバへの利用を容易にするアプリケーションは多く存在する。
    その一方で、サーバーへの通信接続をシステム管理者が容易に制御管理するためのアプリケーションは少ない。
    そのような通信制御のアプリケーションなしでは、システム管理者はユーザーの接続ごとに手動でコマンドを実行し
    管理しなければならず、サーバーを利用するユーザーが多くなるに従いシステム管理者への負担が多くなってしまう。
    \\
    \ そこで、本研究ではサーバーへの通信制御に有効なアプリケーションを調査することを目的とし、3台のサーバーと
    一台のL2スイッチを用いて隔離ネットワークを構成し、ゲートウェイサーバーに個人のソースコード投稿サイトに公開されている
    通信制御のアプリケーションを試用することで、通信制御の容易さ、通信の可視化度合い、セキュリティ、3点について検討を行う。
    
\end{abstract}

\tableofcontents
\clearpage
\chapter{はじめに}

\section{研究背景}
クラウドサービスの普及により、一般ユーザーが遠隔サーバーにアクセスし、その資源を利用する機会が増えてきた。
Googleのクラウドサービス(GCP)、Amazonのクラウドサービス(AWS)、Microsoftのクラウドサービス(Azure)などのように
ユーザーの目的に合わせて、遠隔サーバーの利用を容易にするアプリケーションが多く存在する。



しかし、一方でサーバーへの通信接続をシステム管理者が容易に制御管理するためのアプリケーションは少ない。
そのような通信制御のアプリケーションなしでは、システム管理者はユーザーの接続ごとに手動でコマンドを実行しなければならず、サーバー
を利用するユーザーが多くなるに従いシステム管理者への負担が多くなってしまう。\par
令和2年現在新型コロナウィルス感染症の世界的大流行により在宅勤務(テレワーク)が推奨され、
自宅から社内サーバーへのアクセスを余儀なくされている。そして、同年7月下旬に政府が在宅勤務7割を企業に要請した \cite{covid_nikkei}
ことにより、今後ますますテレワークの人口が増大していきそれに伴い、大規模なリモートアクセスにおけるセキュリティがより重要になり、
社内のシステム管理者の負担がますます増加していくことが予想できる。\par 


%ここで現在の課題や実社会でのネットワーク構成やアクセス図、セキュリティの問題点などを記す。


%社内システム管理者はテレワークによる社外ネットワークからの大規模アクセスのためにも、自社のネットワーク構成を
%見直す必要があるかもしれない。
%\\\\
%また、アプリケーションの適応先で小規模、大規模専用のものと分類し、その中で使われている技術についても解説を行う。
%\\\\

\section{研究目的}
%本研究では、図1のような隔離ネットワークを構成し、そこにソースコード投稿サイトに公開されている通信制御アプリケーションを試用し、
%その中で小規模と大規模アプリケーションいうように分類し検討を行う。


総務省テレワークセキュリティガイドライン \cite{telework_guideline}には、テレワークの利用方法が6パターンに分類されており、それとともに
対策すべき観点もまとめられている。それを以下の表\ref{telework_table}に示す。
下表\ref{telework_table}によると6パターンのうち1、2、6の3パターンが外部から社内サーバにアクセスを必要とする「オンプレミス型」のもので、
3、4、5の3パターンはインターネット上のクラウドサービスを利用している「オフプレミス型」である。

それぞれのパターンの細かな違いは、テレワーク利用端末にデータを保存できるかなどであり、
オンプレミス型もオフプレミス型どちらもインターネットを介して、サーバにアクセスしている。
\par 本研究では
「トンネリング」や「VPN」などのテレワークや遠隔アクセスの根底にある技術の解説を行うと同時に、
社内ネットワークを想定した、隔離ネットワークを構成し、そこにソースコード投稿サイトに公開されている
通信制御アプリケーションを試用することで、
その中で小規模と大規模アプリケーションで分類しソフトウェアの検討を行う。

%そこで使用されている遠隔アクセス技術、認証技術の解説も行う。

本研究でまとめることで、ブラックボックス的にリモートアクセスを行うのではなく、
どのような技術や仕組みを用いて実現しているか、高いセキュリティの実現のためにどのようなプロトコルを用いているかを、
リモートアクセス利用者の理解を深めることを目的とする。
それと同時にシステム管理者が通信接続の制御をすることを容易にするためのアプリケーションを調査する。
%ソースコード投稿サイトに公開されているアプリケーションの中には、商用としている一部を
%オープンソースとして公開しているものも存在する、


\begin{table}[h]
    \centering
    \caption{テレワークの6種類のパターン}
    \includegraphics*[width=1.0\textwidth,page=1]{graphs/telework_list.pdf}
    \label{telework_table}
\end{table}


\begin{figure}[h]
    \centering
    \includegraphics*[width=0.9\textwidth,page=14]{graphs/network_archtecture.pdf}
    \caption{ オフプレミス型}
    \label{cloud_graph}
\end{figure}


\begin{figure}[h]
    \centering
    \includegraphics*[width=0.9\textwidth,page=15]{graphs/network_archtecture.pdf}
    \caption{オンプレミス型}
    \label{onpremise_graph}
\end{figure}




\chapter{遠隔アクセス技術}
本章では遠隔サーバーに接続し操作するための技術やソフトウェアの解説を行う。

\section{暗号化技術}
暗号化技術は、情報の保護やコンピュータセキュリティに欠かせない技術である。
通信内容の保護のために、一般的に次に示す二種類の暗号化技術を使用して、認証及び暗号化通信を行っている。
\begin{itemize}
    \item 共通鍵暗号方式
    \item 公開鍵暗号方式
\end{itemize}
それぞれの暗号方式は様々なアルゴリズムによって実現されるが、元となる「平文」データを「鍵」を使って「暗号文」に変換している。
また、「暗号文」は「鍵」を使用して、「平文」に復号できる。

\subsection{共通鍵暗号方式}
共通鍵暗号方式は、下図\ref{shared_key}のようにAとBで共通の鍵を使用して、暗号化と復号化を行う。
安全にデータを暗号化するために、暗号化通信をする前に共通鍵(秘密鍵)を事前に秘密に共有する必要がある。

共通鍵暗号方式は、後述の公開鍵暗号方式に比べて、演算の処理量が少ないとういう利点ある。
そのため、SSHプロトコルやインターネット上での通信において、通信内容の
暗号化にはこの共通鍵暗号方式を採用している。

代表的な暗号化アルゴリズムに「AES」が存在する。

\afterpage{\clearpage}
\newpage

\begin{figure}[h]
    \centering
    \includegraphics[width=0.9\textwidth, page=8]{graphs/network_archtecture.pdf}
    \caption{共通鍵暗号方式での暗号化}
    \label{shared_key}
\end{figure}

\begin{figure}[h]
    \centering
    \includegraphics[width=0.9\textwidth, page=9]{graphs/network_archtecture.pdf}
    \caption{公開鍵暗号方式での暗号化}
    \label{public_key}
\end{figure}


\afterpage{\clearpage}
\newpage
\subsection{公開鍵暗号方式}
%共通鍵暗号方式においては、送信者と受信者が同じ共通鍵を共有しておく必要があった。
%したがって、鍵を
公開鍵暗号方式は、二種類の鍵である公開鍵と秘密鍵をペアで使用する。
公開鍵は暗号化、秘密鍵は復号化の際に使用される。暗号鍵である公開鍵は公開され、
復号鍵である秘密鍵は安全に保管する必要があり、秘密鍵は公開鍵から求めることはできない性質を持つ。
こうすることで、誰もが暗号文を生成できるが、復号は秘密鍵を持つ人しかできないという状況が実現できる。
\if0
この公開鍵と秘密鍵には次に示す性質があり、公開鍵暗号方式はこれら
の性質を利用して暗号化や署名を実現している。

\begin{itemize}
    \item 公開鍵で暗号化した平文は、秘密鍵で復号できる
    \item 公開鍵で暗号化した平文は、公開鍵では復号できない
    \item 秘密鍵で暗号化したデータは、公開鍵で復号できる
    \item 公開鍵から秘密鍵を生成できない。
\end{itemize}
\fi

ここで、公開鍵暗号方式での暗号化について上図\ref{public_key}に示す。この図では、鍵ペアを作成したBが、公開鍵をAに公開してる。Aは、Bの公開鍵を使用して
平文を暗号化して、Bへ送付する。送付された暗号文はB自身の秘密鍵でのみ復号できる。

SSHプロトコルやインターネット上での通信において、共通鍵暗号方式を利用するために、共通鍵を
事前に公開鍵暗号方式を使って、2者間で共有している。
代表的な暗号化アルゴリズムに「RSA暗号」「Diffi-Hellman鍵共有」等がある。

\section{トンネリング(Tunneling)}\label{Tunnel}
ここでは、トンネリングの概要と\ref{aboutVPN}節に関係するVPNトンネリングプロトコルの説明を行う。

\subsubsection*{トンネリングとは}
トンネリングとは、「物理的、論理的に離れた2点間に、閉じられた仮想的な回線を作成すること」である。
%「パケットやフレームを他のパケットやフレームの中にカプセル化すること」である。
一般的に、パケットやフレームを他のパケットやフレームの中にカプセル化することで実現される。
これにより、プライベートアドレスの隠蔽や、カプセル化後に暗号化をすることでデータの保護を行うことができる。
後に説明するVPNにおいてもトンネリング技術は必要不可欠なものとなっている。
%トンネリングはVPNにおいてとても大切な役割を持つが、
ただし、
「トンネリング=VPN」であるわけではないというのは重要な点である。
%トンネルはVPNであるというわけではなく、VPNはトンネルであるというわけではないというのはとても重要な点である。

\par トンネリングの機能を以下に示す。
\begin{itemize}
    \item データ転送の容易化 \mbox{}\\
    トンネリングには、パケットまたはフレーム全体を直接特定の場所に転送する、「宛先を切り替える」
    機能を持つ。これにより、それらが宛先ネットワークに到着するとそのネットワークを管理している組織のセキュリティや
    ネットワークポリシーによる制御を受けることができる。
    \item 組み込みセキュリティ機能\mbox{}\\
    トンネリングプロトコルの中では、追加セキュリティ(暗号化、認証など)がプロトコルに組み込まれているもの
    がある。そのようなプロトコル(プロトコル群)の代表例として「IPSec」、「L2TP」や「PPTP」がある。
\end{itemize}


\subsection{IPSec}
IPSec(IP Security Architecture)は、IPパケットのデータの完全性(Integrity)と機密性(Confidentiality)を
保証する機能を
持っているプロトコル(プロトコルの集合)であり、ネットワーク層で動作しIP上(インターネットプロトコル)でしか使用できない。
これらの機能を提供するための方法として、IPSecは以下の3つの要素を持ち、「AH」、「ESP」、「IKE」などの
複数のプロトコルから構成されている。
\begin{itemize}
    \item 認証(ユーザレベルでなく、パケットレベルの)\mbox{}\\
    データの送信者が間違いなく本人であること、及び送信されたデータが受信されたデータと同一であることを
    検証する動作である。\par
    実現のためのプロトコル: AH、ESP
    \item 暗号化\mbox{}\\
    適切な鍵を持たない人にはデータを読み取ることができないようにする動作である。\par
    実現のためのプロトコル: ESP
    \item 鍵管理\mbox{}\\
    送信者と受信者の間でセキュリティ「鍵」の値を一致させたり、取り決めさせたりする動作である。\par
    実現のためのプロトコル: IKE
\end{itemize}
%IPSecにおけるパケットレベルの認証にはIPSecのAuthentication Header (認証ヘッダー) が使用される。

%\subsubsection*{IPSecを構成する2つのプロトコルと鍵交換のプロトコル}
\subsubsection{IPSecを構成するプロトコル}

\begin{itemize}
    \item AH(Authentication Header)\mbox{}\par
    パケットが改ざんされていないかどうか認証を行うが、パケットの暗号化はできない。\\RFC2402 \cite{RFC2402}で定義されている

    \item ESP(Encapsulated Security Payload)\mbox{}\\
    パケットが改ざんされていないかどうか認証を行い、
    パケットのペイロード部の暗号化(DES or 3DES or AES)を行う。\\RFC2406 \cite{RFC2406}で定義されている。 
    \item IKE(Inernet Key Exchange)\mbox{}\\
    秘密鍵情報の交換を安全に行う。(Difiie-Hellman鍵共有などを用いる)
    \\RFC2409 \cite{RFC2409}で定義されている。
\end{itemize}

%\subsubsection*{IPSecの動作}
%IPSecは送信側と受信側というように、常に二組のユーザが関わりを持つ。IPSecの用語では
%このような関係をSA(Security Association)と呼ぶ。
IPSecでは、事前に通信を行うホスト同士が認証・暗号化のアルゴリズムや暗号鍵を共有するコネクションを形成
する必要がある。この関係をSA(Security Association)という。\par
IPSecに対応したホストやファイアウォールではデータベース内に複数のSAを保有することができ、あるIPSec
のパケットがどのSAに対応しているかは、IPSecのヘッダ中に含まれるSPI(Security Parameters Index)というポインタに指定される。
\par 
\subsubsection*{通信モード}
IPSecは通信モードが2つあり、「トンネルモード」と「トランスポートモード」である。
ここでは2つのモードを簡単に説明する。

まず、「トンネルモード」では、
    それぞれ別のネットワークとして構成されているネットワークで発生する通信パケットを、ゲートウェイでカプセル化して
    転送する。つまり、元のIPパケットをゲートウェイで「外側」のIPパケットの内部にカプセル化し、
    元のパケットをペイロード(正味のデータ部分)としてまったく新しいパケットを作成し転送している。
    その後ゲートウェイのIPに従って経路制御され、通信相手のゲートウェイでカプセル化されたパケットが取り出され、
    宛先となっているネットワークに配送される。

次に「トランスポートモード」は、
端末間でのIPSec通信を実現するために、転送する前に元のパケットのペイロードだけを暗号化する。
このとき、IPヘッダは暗号化されず、そのまま経路制御のために利用される。


\subsection{PPTP}
PPTP(Point-to-Point Tunneling Protocol)\cite{RFC2637}はデータリンク層で動作するトンネリングプロトコルの一つである。
リモートクライアントからプライベートな企業のサーバーにデータを安全に伝送することをようにするための手段として
設計された。サーバ/クライアントモデルで、データリンク層でカプセル化を行う。最も古いVPNプロトコルの一つであるが
設定が最も容易で計算速度が速いため、古く遅いデバイスで使用される。
深刻なセキュリティ上の脆弱性が存在するため、2020年現在使用することは勧められていない。
\subsubsection*{PPTPの通信手順}
以下にPPTPの通信手順を簡単に説明する。
\begin{enumerate}
    \setlength{\parskip}{0.05cm} % 段落間
    \setlength{\itemsep}{0.05cm}
    \item 制御用コネクションの確立(PPTPトンネルのための準備)\mbox{}\\
    トンネルの構築、維持、切断処理に関する制御情報をやり取りするために、トンネル以外に制御用
    情報の交換のためのコネクションを確立することである。制御コネクションにはTCPを利用している。
    \item PPPセッションを利用してPPTPトンネルを構築\mbox{}\\
    ここでは、PPP(Point-to-Point Protocol)\cite{RFC1661}が本来持っている仕組みを利用して、PPTPトンネルを構築するための
    認証及び暗号化の方法を選んで、実際に認証と鍵交換を行う。
\end{enumerate}

\subsection{L2TP}
L2TP(Layer 2 Tunneling Protocol)\cite{RFC2661}はL2F(Layer 2 Forwarding Protocol)\cite{RFC2341}とPPTPのアップグレードとして設計された。
VPNに利用することができるが、L2TPそのものは暗号化の仕組みを持っていないため、IPSecの暗号化機能と組み合わせて
VPNの暗号化通信を実現している。
L2TPはIPSecとは異なり、実装次第でトンネルするネットワークが必ずIPネットワークである必要がない。
そして、重要な点としてL2TPではカプセル化されたデータはUDPのデータとして(L2TPのメッセージとして)IPネットワーク上を配送
される。
\subsubsection*{L2TPの通信手順}
L2TPでは以下のような手順で通信経路の確立を行う。
\begin{enumerate}
    \setlength{\parskip}{0.05cm} % 段落間
    \setlength{\itemsep}{0.05cm}
    \item 初めに制御コネクションを確立してトンネルを構築する。
    \item その後同じトンネルを使ってユーザセッションを確立する。
    \item 最後にPPPによるリンクを確立する。
\end{enumerate}


\section{Telnet}
Telnet \cite{RFC97}は、ネットワークに接続された機器を遠隔操作するために使用するアプリケーション層の技術である。
サーバ/クライアント方式で提供され、Telnetサーバが操作される側、クライアントが操作する側で動作する。
Telnetを使うことでオフィスのデスクにいながら、マシンルームにあるサーバ、ルータ等の機器をパソコン上で操作できる。
サーバにはtelnetクライアント、ルータなどの機器にはtelnetサーバーのサービスが有効であることが前提である。
基本的にはポート番号23を使用する。


\subsection{Telnetの仕組み、使用法}
Telnetの接続までの流れは以下の図\ref{telnet_flow}のようになっている。
クライアントからのTelnetはコマンドプロンプトやTerminalから、「telnet 150.65.136.94」というように
Telnetコマンドと接続したいサーバのIPアドレス
を入力するか、WindowsではTera Term等でIPアドレスと入力してTelnet接続を行う。
TCPによるコネクション確立後クライアントのコマンドプロンプトでTelnetサーバから応答画面が表示される。
Telnetで遠隔操作を行うためには対象の機器にログインする必要があるため、最初の応答画面ではパスワード要求がされる。

\begin{figure}[H]
    \centering
    \includegraphics*[width=1.0\textwidth,page=1]{graphs/network_archtecture.pdf}
    \caption{Telnet接続フロー}
    \label{telnet_flow}
\end{figure}

問題点として、認証も含めすべての通信を暗号化せずに平文のまま送信するため、パスワードを盗むのは比較的容易である。
同様の機能を持ち、情報を暗号して送信することができるSSHが存在し、セキュリティの観点からTelnetよりもSSHが推奨されている。

\section{SSH}
上記のTelnet は、遠隔操作するサーバーの認証情報を含め、通信を暗号化せずに平文のまま通信を行う。その結果、通信内容を盗聴されると
認証情報や通信内容が簡単に盗まれてしまう危険性がある。
この問題を解決してくれるのが、Telnetと同様な機能を持ちかつ通信内容を暗号化してくれる「SSH」(Secure SHell) \cite{RFC4253}である。
SSHには「SSHv1」と「SSHv2」の二種類のバージョンが存在し、本論文では安全性の高い「SSHv2」のみの解説を行う。

\begin{figure}[h]
    \centering
    \includegraphics[width=1.0\textwidth, page=3]{graphs/network_archtecture.pdf}
    \caption{telnet接続による脅威}
    \label{telnet_flow}
\end{figure}
\begin{figure}[h]
    \centering
    \includegraphics[width=1.0\textwidth, page=4]{graphs/network_archtecture.pdf}
    \caption{SSH接続によるセキュアな運用管理}
    \label{SSH_security}
\end{figure}


\afterpage{\clearpage}
\newpage

SSHの技術を用いるために、「OpenSSH」というソフトウェアが一般的に使用される。
認証方式として大きく2つあり、「パスワード認証」と「公開鍵認証」が使われる。
「パスワード認証」は、ログイン時に利用するアカウント情報をそのまま異利用し、IDとパスワードが一致すれば認証を行う。
「公開鍵認証」は、クライアントが公開鍵と秘密鍵を生成し、クライアント側が持つ秘密鍵が、サーバー側が持つ公開鍵に対応するものであるか
どうかで認証する。クライアント側が持つ秘密鍵はネットワーク上に送信されることはなく、サーバ側が持つ公開鍵から秘密鍵を推測されないため、
パスワード認証よりも安全な認証を行える。
SSHの公開鍵とユーザIDが"\textasciitilde/.ssh/known\_host''ファイルに登録される。
\subsection{SSHの機能}



%---------------メモ----------------
\begin{enumerate}
    \item セキュアリモートログイン(ssh)\mbox{}\\通常、
    Secure Shell(SSH)と呼ばれる機能のこと。セキュアリモートログインを使用すると、インターネット経由でも安全に、
    運用端末からSSHサーバーへログインできる。また、通信内容を他者に見られないため、安全な運用管理を実現できる。
    \\使用コマンド例   \$ ssh pc15@150.65.136.94

    \item セキュアコピー(scp)\mbox{}\\セキュアコピーを使用すると、SSHサーバからファイル転送を受け取ることできる。
    また、通信内容を他者に見られたり、改ざんされたりすることがないため、安全な運用管理を実現できる。
    \\使用コマンド例   \$ scp -r pc15@150.65.136.94:/home/pc15/画像 /Users/shuto/desktop\\
    これにより、sshサーバから、画像ディレクトリをファイル転送を受け取れた。

    \item セキュアFTP(sftp)\mbox{}\\
    「SSH」で暗号化された通信路を使ってFTPを使用している。
    セキュアFTPを使用すると、SSHサーバーにファイルを転送することができる。
    セキュアコピーと同様に、通信内容の盗聴や、改ざんを防ぐことができる上、セキュアコピーではできないファイルの
    名前変更や削除などが可能である。
    
\end{enumerate}

%更にログインしなくてもサーバのコマンドを実行できるSSHサーバへログインするためのユーザの認証方法には、
%telnetで使用されていたパスワード認証の他に、より安全な公開鍵認証を使用できる。
%公開鍵認証を使用することで、パスワードが漏洩し、他者に利用されることを防ぐ。


\subsection{SSH接続までの流れ}

まず初めに暗号化通信路の確立までの流れを下図\ref{SSH_flow}のように示す。
\begin{figure}[h]
    \centering
    \includegraphics[width=1.0\textwidth, page=5]{graphs/network_archtecture.pdf}
    \caption{SSH接続確立までのフロー}
    \label{SSH_flow}
\end{figure}

%ーーーーーーーーーー以下メモーーーーーーーーーーーー


\begin{enumerate}[(a)]
    \setlength{\parskip}{0.05cm} % 段落間
    \setlength{\itemsep}{0.05cm}
    \item バージョンと各種暗号方式の交換\mbox{}\\ 
    初めに、サーバとクライアントの間で使用するSSHバージョン情報を交換し、
    SSHv1、SSHv2のどちらを利用するかを決定する。
    その後、使用できる鍵交換方式、希望する公開鍵暗号方式、共通鍵暗号方式、
    メッセージ認証コード、のアルゴリズムの各リストを交換する。
    \item ホスト認証と暗号化通信路の確立\mbox{}\\
    各SSHサーバーは、それぞれ異なるホスト鍵ペア(ホスト公開鍵とホスト秘密鍵)を保持している。ホスト鍵ペアはインストール時に生成される。
    クライアントは、サーバの認証、識別のためにこれらの鍵を使用する。
    サーバ及びクライアントは、交換した共通鍵暗号方式やメッセージ認証コードのリストから、使用するアルゴリズムを決定する。その後、
    Diffi-Hellman鍵交換方式で、暗号化通信路に使用する共通鍵を交換する。共通鍵の交換中にサーバの持つホスト公開鍵をクライアントで保持している
    ホスト公開鍵のデータベースと照合して、サーバの認証も行う。
    ここで、Diffi-Hellman鍵交換方式は、交換する鍵を直接送ることなく、両者で鍵を共有できるアルゴリズムである。
    \item ユーザ認証\mbox{}\\ホスト認証後、暗号化通信路が確立されると、公開鍵暗号方式またはローカルパスワードによるユーザ認証を行う。
    \begin{enumerate}[(1)]
        \setlength{\parskip}{0.05cm} % 段落間
        \setlength{\itemsep}{0.05cm}
        \item 公開鍵暗号方式によるユーザ認証\mbox{}\\ 接続先のサーバにはあらかじめユーザの公開鍵を登録しておく。クライアントでは、
        登録されているユーザ公開鍵に対応した、ユーザが所持している秘密鍵を使用して認証する。
        SSHv2では、「電子署名」という方法を使用する。\\
        \ \ まず、クライアントでは、ユーザ名、ユーザの公開鍵、ユーザの公開鍵アルゴリズムを記述した認証要求メッセージを作成する。
        そして、作成した認証要求メッセージに対して、ユーザの秘密鍵を使用して電子署名を作成する。最後にサーバーに対して認証要求メッセージに
        最後に、サーバに対して、認証要求メッセージに電子署名をつけたものを送付する。
        
        \ \ サーバでは、送付された認証要求メッセージから、ユーザ名とユーザ公開鍵を取り出し、登録済みのユーザとユーザの公開鍵であることを確認する。
        また、登録されているユーザの公開鍵を使用して、送付された電子署名を審査し、正しいユーザの電子署名であることを確認できると、ユーザの認証成功となる。
        \item ローカルパスワードによるユーザ認証\mbox{}\\
        Telnetと同様に、サーバでローカルに設定されたパスワードを使用してユーザ認証を行う。しかしパスワードは暗号化された通信路を経由するため、
        第三者には見えない
        
    \end{enumerate}
    \item ログイン後\mbox{}\\
    ユーザ認証に成功すると、セッションが確立し、ユーザはログインする。ここで通常はターミナルのセッションが開始される。
    
\end{enumerate}


\section{VPN}\label{aboutVPN}

\subsection*{VPNとは}
VPNとは、「Virtual Private Network」の略であり、「仮想専用線」や「仮想閉域網」と訳され、通信事業者のネットワークやインターネットなどの公衆ネットワーク上で作られる、仮想的な専用ネットワークの
総称である。
VPNを利用することで、公衆回線(インターネット)での脅威を防ぎ、安全なリモートアクセスを実現することができる。
VPNの実現のための主な方法として、ここでは「IPSec-VPN」と「SSL-VPN」を挙げる。
まず初めに下表\ref{IPsec_SSL_diference}に2つの違いを簡単にまとめ、その後各々の説明を行う。
\begin{table}[h]
    \centering
    \caption{IPSec-VPNとSSL-VPNの違い}
    \includegraphics[width=1.0\textwidth, page=2]{graphs/telework_list.pdf}
    \label{IPsec_SSL_diference}
\end{table}

\subsection{IPSec-VPN}
IP-VPNとは、\ref{Tunnel}節で説明したIPSecの技術を利用してVPNを構成する方法である。
IPSec-VPNでは、VPNゲートウェイ装置との間にVPNトンネルを作るため、リモートアクセス端末に専用のソフトをインストールする
必要がある。また、暗号化や認証のための設定などの環境設定を事前に行う必要があり、
IPSec-VPNは、複数のオフィス間でVPNを張る等、固定的なエンドポイント同士でのVPN接続に向いている。

モバイルデバイスと社内システムをVPNで接続するなどには、より手軽に利用できるSSL-VPNを使用するのが一般的である。


\subsection{SSL-VPN}
SSL-VPNとは、暗号化にSSL技術を使用したVPNである。
IPSec-VPNはクライアントPCに必ずVPNのクライアント用ソフトウェアをインストールする
必要があるのに対して、SSL-VPNの場合はWebブラウザさえあれば通信可能である。

SSL-VPNには、「リバースプロキシ」、「ポートフォワーディング」、「L2フォワーディング」の3つの方式がある。
まずはじめに3つの方式の比較を簡単に下表\ref{SSLVPN_impliment_table}に示した後、それぞれを簡潔に説明する。

\begin{table}[h]
    \centering
    \caption{SSL-VPNの実装方式の比較}
    \includegraphics[width=1.0\textwidth, page=3]{graphs/telework_list.pdf}
    \label{SSLVPN_impliment_table}
\end{table}

\subsubsection*{SSL-VPN(リバースプロキシ方式)}

「リバースプロキシ」は、通常のクライアント側(利用者側)の「プロキシ」ではなく、サーバー側(アクセスされる側)に
位置するものである。簡単な構成を下図\ref{reverse_proxy}に示す。

\begin{figure}[h]
    \centering
    \includegraphics[width=1.0\textwidth, page=16]{graphs/network_archtecture.pdf}
    \caption{リバースプロキシの図}
    \label{reverse_proxy}
\end{figure}

クライアントからはWebサーバにアクセスしているように見えるが、実際は
SSL-VPN Gateway(リバースプロキシ)が仲介者としてClientのRequestに沿って各ローカルのWebサーバにアクセスしている。

これにより、Webサーバに外部からの直接的なアクセスができなくなり、改ざんや不正侵入などのリスクを減らすことができる
上にSSL-VPN Gateway にファイアウォールや認証機能を追加することでセキュリティの強化を行うことができる。
また、SSL-VPN GatewayだけにSSL証明書を適用すればよくなる。

リバースプロキシ方式は、Webブラウザで動作しないアプリケーションは使用できないという問題があった。
そこでリモートアクセスしてくるクライアントにJavaやActiveXで作成されたモジュールを追加させた上でSSL-VPN通信を
行う方法が考えられた。それが、「ポートフォワーディング」と「L2フォワーディング」である。

\subsubsection*{SSL-VPN(ポートフォワーディング方式)}

一般的に上記のリバースプロキシ方式では、SSL-VPN Gatewayにファイアウォールを設置し、外部から利用できる
アプリケーションを制御(Webサーバは閲覧できるが、Telnetはできないなど)している。その制御には、通信されるデータに含まれる、ポート番号と呼ばれるアプリケーション
の種類を示す情報を使用している。
\begin{figure}[h]
    \centering
    \includegraphics[width=1.0\textwidth, page=17]{graphs/network_archtecture.pdf}
    \caption{ポートフォワーディングの図}
    \label{portfowarding}
\end{figure}

ポートフォワーディングとは、ファイアーウォールを通過できないアプリケーションのデータのポート番号を、
通過できるアプリケーションのポート番号に変換することにより、ローカルネットワークと
グローバルネットワーク(インターネット)との通信を可能にする機能である。
ポートフォワーディング方式では、この機能を用いて任意のアプリケーションの通信を
上図\ref{portfowarding}のようにHTTPSのポート番号に
変換し、ファイアーウォールを通過させることでSSL-VPNを実現する。

\subsubsection*{SSL-VPN(L2フォワーディング方式)}
L2フォワーディング方式ではクライアントPCにSSL-VPNクライアントソフトをインストールする。

L2フォワーディングでは、アプリケーションのデータをHTTPパケットでカプセル化してSSL通信を行う。
ポートフォワーディング方式のように、SSL-VPN Gateway で通信するサーバのIPアドレス/ポート番号を事前に
定義する必要がないため、幅広いアプリケーションをサポートすることが可能である。


専用のソフトウェアをインストールすると、クライアントにSSL-VPN用の仮想NICが追加され、仮想NIC経由の通信がすべて
SSLで暗号化される。仮想NICのIPアドレスは、通常SSL-VPN接続とともに自動設定される。そして、クライアントは仮想NICによって
社内ネットワークに直接接続されているのと同等に扱うことができるようになる。

\begin{figure}[h]
    \centering
    \includegraphics[width=1.0\textwidth, page=18]{graphs/network_archtecture.pdf}
    \caption{L2フォワーディングの図}
    \label{L2fowarding}
\end{figure}


\par 企業においてSSL-VPNを用いてテレワークなどのために社内ネットワークに遠隔アクセスする場合、ログアウト後に
そのクライアントPC(テレワーク利用端末)にデータが残らないよう自動的にデータが削除されるように
実装させるのが一般的である。

参考として、CiscoのVPNゲートウェイ装置でSSL-VPNを行う場合、「リバースプロキシ方式」、
「L2ポートフォワーディング方式」が利用できる。「ポートフォワーディング方式」は利用できない。


%VPNを使用する目的は大きく2つある。安く通信内容の漏洩を防ぐことと、ある程度の通信品質(QoS)を確保することである。


\chapter{認証技術}\label{Authentication}
ここでは認証技術として代表的な技術である、「LDAP」、「Kerberos」、「RADIUS」、「Active Directory」の解説を行う。

\section{LDAP}
\subsubsection*{概要}
LDAP (Lightweight Directory Access Protocol) \cite{RFC4511}は、Active Directoryのようなディレクトリサービスにアクセスするためのプロトコルである。
%ディレクトリサービスというユーザやコンピュータといった情報を管理するサービス
LDAP自体はプロトコルであり、サービスやシステムを指すものではない。
LDAP認証は、ユーザの名前とパスワードの組み合わせの検証にのみ使用される。そのため、LDAPを利用して認証を行うようにする前に、ユーザ情報
はデータベースに存在しなければならない。

代表的なLDAPを利用できるソフトウェアに、「Open LDAP」、「Active Directory」が存在する。
%LDAPを実装したデータベースをLDAPサーバと呼び、代表的なものに「Open LDAP」、「Active Directory」が存在する。\par
\subsubsection*{ディレクトリサービス}
ディレクトリサービスとは、ディレクトリと呼ばれるデータベースから、ユーザ名やマシン名などのキーを元にデータを検索、参照するための
サービスである。
一般的にデータベースと呼ばれる、SQL言語等を用いて扱う「RDB (Relational Data Base)」では、データ間の関係性を利用して、
データの参照、挿入、更新、削除、といった操作を行うため、それらは少し異なるものである。
グローバルでサービスを提供する場合には、分散型ディレクトリサービスが用いられ、DNS(Domain Name Service)が分散型ディレクトリサービス
として有名である。

\subsubsection*{LDAP、ディレクトリサービスの特徴}
%LDAPとはデータ追加や削除よりも検索を重視したプロトコルであるため
%、顧客、商品情報管理のように頻繁に更新されるデータを扱うのは能力は高くない。

\begin{itemize}
    \item 読み取りが高速
    \item 分散型の情報格納モデル
    \item 高度な検索機能をもつ
\end{itemize}

LDAPの特性としては、「情報の参照、検索」に特化している。
このようになる理由としては、
ディレクトリサービスとして利用されるものは、一般的なデータベースのように読み取りと書き込みが同じ頻度で発生することはなく、
大規模システムではユーザ情報の利用は参照検索が最も頻繁に起こるため、それらの操作に対する高い性能が必須であるため、
このような特性となっている。

LDAPを用いたデータの集中管理の様子を下図\ref{LDAP_data_manage}に示す。
\begin{figure}[h]
    \centering
    \includegraphics[width=1.0\textwidth, page=10]{graphs/network_archtecture.pdf}
    \caption{LDAPを用いてデータを集中管理の図}
    \label{LDAP_data_manage}
\end{figure}

%LDAPを利用することで様々なサービスからLDAPかで格納された情報を参照可能なため、提供するすべてのサービスで単一のユーザー情報を
%持ちにユーザー
%認証が可能になる。(SSO シングルサインオン)

\subsubsection*{LDAPできること}
\begin{enumerate}
    \setlength{\parskip}{0.05cm} % 段落間
    \setlength{\itemsep}{0.05cm}
    \item リソースの一元管理\mbox{}\\多数のクライアントがある場合、1台1台にIDパスワード情報を入れることなく、LDAPサーバー1台
だけ登録すれば、どのクライアントからも同じIDパスワードでログインできるようになり、さらに環境もログイン時に取得できるようになる。


    \item リソースのアクセス制御\mbox{}\\特定のIPアドレスからであれば、読みと書き可能であるが、それ以外からは読みしかできない。
    \item 各種サービスとの連携\mbox{}\\多くのアプリケーション(Open Source Software)と連携することができる。初期のユーザ情報作成
をLDAPデータベースから行い、認証をLDAPサーバに委譲することができる。これの発展形として、一つのサーバで認証すれば、他のサーバでは
認証無しでログインできる仕組みである、シングルサインオン(SSO)を実現できる。
\end{enumerate}

%\subsubsection*{LDAPの仕組み}
%-----------ここは図を使って説明できたら良い。

\section{Kerberos}
\if0ネットワーク内でのシステムの安全性と利便性は共存しにくいことがある。
単純にどのサービスがネットワーク上で稼働しているか、そして、使用されているサービスの動作を管理者が追跡するだけでも膨大な時間がかかることがある。
さらに、FTPプロトコルやTelnetプロトコルのようにデータを暗号化せずにネットワーク上でパスワードを送信させるというような、プロトコル自体が安全でないとき
ネットワークサービスへのユーザ認証は危険を伴うことになる。
\fi

\subsubsection*{Kerberosとは}
Kerberos \cite{RFC4120}とは、ネットワーク上でユーザの認証を行う方式の一つであり、サーバとクライアント間の身元確認のために使用される。
クライアント/サーバ間の通信を暗号化でき、比較的セキュリティが強固な認証方式となっている。
ケルベロス認証では一度ログインすると、
「チケット」と呼ばれるものを用いて認証を行えるようになるため、
次回のログイン時にID・パスワードを改めて入力する必要がなくなり、シングルサインオン(SSO)を実現できる。\par 
利用例としては、Active Directory のユーザ認証の際に用いられている。名前はギリシャ神話の地獄の門を守る番犬ケルベロスに由来している。

\subsubsection*{Kerberosの構成要素}
Kerberosの仕組みを解説する前に、用語「KDC、AS、TGS、プリンシパル、レルム」の説明を行う。
\begin{itemize}
    \setlength{\parskip}{0.1cm} % 段落間
    \setlength{\itemsep}{0.1cm} 
    \item KDC (Key Distribution Center)\mbox{}\\サーバとユーザに関する信頼関係の情報を一括管理する中央データベース。これをLDAPサーバにすることもできる。
    \item AS (Authentication Server) \mbox{}\\認証サーバで、ユーザからの認証を受け付けるサーバ。
    \item TGS (Ticket Granting Server) \mbox{}\\チケット発行サーバ。各サーバを利用するためのチケットを発行するサーバ。
    \item プリンシパル (principal) \mbox{}\\ KDC認証を行うユーザやサーバのこと。
    \item レルム (realm)\mbox{}\\同じKDCの配下にあるシステムをグループとして定義する論理ネットワーク。
\end{itemize}
これらを構成すると以下の図\ref{KerberosCompornent}になる。
%------------ここにkerberos の構成図を載っける。ここはKerberos認証のコンポーネント---------
\begin{figure}[h]
    \begin{center}
        \includegraphics[width=1.0\textwidth, page=11]{graphs/network_archtecture.pdf}
        \caption{Kerberos認証の構成要素}
        \label{KerberosCompornent}
    \end{center}
\end{figure}\\
\subsubsection*{Kerberosの認証の仕組み}
Kerberos認証では、ユーザが正しいユーザIDとパスワードをAS(Authentication Server)に送信して認証に成功するとTGS(Ticket Granting server)から
チケットと呼ばれるデータを受け取れる。Kerberos認証ではこのチケットを認証に使用する。サーバはアクセスしてくるユーザがアクセス権を持っている
かどうかをユーザIDとパスワードではなくチケット(クライアントID、タイムスタンプ、有効期限が記されている)を使用して確認する。
認証時にチケットを使用することでアカウント(ユーザID、パスワード)の漏洩を防いでいる。全体的な流れを
以下図\ref{KerberosAuthority}に示す。
%https://milestone-of-se.nesuke.com/sv-advanced/activedirectory/kerberos-spnego/
\begin{figure}[h]
    \begin{flushleft}
        \includegraphics[width=1.0\textwidth, page=12]{graphs/network_archtecture.pdf}
        \caption{Kerberos認証流れ}
        \label{KerberosAuthority}
    \end{flushleft}
\end{figure}\par
Kerberos認証では、チケットの盗聴によるなりすましを防ぐために、時刻同期の仕組みが用意されている。
チケットの中にはタイムスタンプ(送信時刻)が記録されており、チケットを受信したサーバがチケットのタイムスタンプとサーバの
持つ時刻と5分以上のズレがあると認証に失敗するようになっている。
したがって、NTP(Network Time Protocol)を使用して、チケット発行側の時刻とチケット利用側の時刻が同じにする必要がある。
\clearpage

\section{RADIUS}

\subsubsection*{RADIUSとは}
RADIUS (Remote Authentication Dial In User Service) \cite{RFC2865}は、ネットワーク上のユーザ認証プロトコルである。
%無線LANや有線LANでのネットワーク接続時のユーザ認証のプロトコルとしても利用されている。
RADIUSによる認証システムは、「RADIUSサーバ」、「RADIUSクライアント」、「ユーザ」の3つの要素で構成されている。

インターネットが普及を始めたころ、ユーザがインターネットへのアクセスするには電話回線を使ったダイヤル
アップが主流であったため、RADIUSサーバはダイヤルアップサービス用の認証サーバとして開発された。
その後ISPや企業などで利用されるようになり、ネットワークが光回線に置き換わった現在でも、ISPの認証サービスなどではRADIUSサーバが継続して
使用されている。また、近年では、Wi-Fiアクセスポイントでの認証などでも、RADIUSサーバが使われている。

\begin{itemize}
    \setlength{\parskip}{0.05cm} % 段落間
    \setlength{\itemsep}{0.05cm} 
    \item RADIUSクライアント\mbox{}\\
    アクセスしてくるユーザの認証要求を受け付けてRADIUSサーバーにその情報を転送する役割。
    NAS(Network Access Server)とも呼ばれる。
    \item RADIUSサーバ\mbox{}\\
    認証要求に応じて認証を実行してアクセスを許可するかどうかを決定する役割。
\end{itemize}

\subsubsection*{RADIUS認証の流れ}
ユーザがネットワークやネットワーク機器を利用したい場合、
全体の認証の流れは下図\ref{RADIUS_Authentication}のようになっている。
\begin{figure}[h]
    \begin{center}
        \includegraphics[width=0.9\textwidth, page=19]{graphs/network_archtecture.pdf}
        \caption{RADIUS認証の流れ}
        \label{RADIUS_Authentication}
    \end{center}
\end{figure}
\begin{enumerate}
    \setlength{\parskip}{0.1cm} % 段落間
    \setlength{\itemsep}{0.1cm} 

    \item ユーザがネットワークに接続しようとする。
    \item RADIUSクライアントは、ユーザに認証を要求する
    \item ユーザがユーザ名、パスワードを入力し、認証を行う。
    \item RADIUSクライアントは、受け取ったユーザ名、パスワードを使って、RADIUSサーバにアクセス認証のRequestを出す。
    \item RADIUSサーバは認証処理を行い、RADIUSクライアントに認証可否を伝える。
    \item RADIUSクライアントは、ユーザに認証結果を伝える。

\end{enumerate}


%-----------------------もしできたら、RADIUSとLDAPの連携の図をさしこみたい----------------------------


\if0
PCからRADIUSクライアントに送信されたユーザ名とパスワードは、RADIUSクライアントからRADIUSサーバへ"Access-Request"メッセージ
として送信される。RADIUSサーバは送信されてきた情報と、RADIUSサーバが保持している情報とを照らし合わせて、正規
ユーザかどうかを識別する。正規ユーザと判断されるとRADIUSサーバは"Access-Request"メッセージでRADIUSクライアントに伝えて、その情報が
ユーザに伝えられる。
\\
\fi

\section{Active Directory}\label{Active Directory}
\subsubsection*{Active Directoryとは}
マイクロソフトによって開発されたディレクトリサービスシステム\cite{active}で、一般的にWindowsOSで使用される。
Active Directoryは複数のサービスの総称であり、「Windowsシステムで認証を行う機能」を持つ。

社内システムにおいて、利用権限によって一部の社員にしかアクセスさせたくないシステムに対して、
AcitiveDirectoryの認証機能を使用して、アクセスの制限を行う\cite{activedirectory}。
\if0
社内システムの中にはアクセスを制限して一部の社員にしかアクセスさせたくないシステムもある。
このようなシステムにアクセス可能かどうかを判断するために、Active Directoryの認証機能\cite{activedirectorydomainservices}を使用してアクセスの制限を
行う。
\fi
\subsubsection*{Active Directoryの機能}

Active Directoryは5つのサービスから構成されている。

\begin{itemize}

    \item  Active Directoryドメインサービス (AD DS)\mbox{}\\
    ユーザーやコンピュータの認証や、管理者が情報を安全管理したり、ユーザがファイルやディレクトリ などのリソースを
    簡単に検索することができる機能である\cite{activedirectorydomainservices}。\\
    一般的にActive Directoryというと、このドメインサービスのみを指すことが多い。
    
    \item Active Directoryライトウェイトディレクトリサービス(AD LDS)\mbox{}\\
    AD DSの簡易版という立ち位置。AD DSの構成要素のうち「データベースの仕組み」、データの検索記録が行えるように
    なったもの。認証を行う機能はない。

    \item Active Directory証明書サービス(AD CS)\mbox{}\\
    公開鍵暗号基盤(PKI)を構築するための、証明書の作成と管理を行う証明機関 (Certification Authority:CA)を作成するサービス。
    
    \item Active Directory Rights Management Services (AD RMS)\mbox{}\\
    ドキュメントの権限管理やコンテンツ保護など、不正使用から情報を保護するための機能。
    AD RMSを利用することで、メールやドキュメントの保護が可能になり、保護されたデータは暗号化される。

    \item Active Directoryフェデレーションサービス(AD FS)\mbox{}\\
    複数のWebアプリケーション間の認証や、異なる組織間での認証など、組織の違いを超えて認証の仕組みを連携する機能。

\end{itemize}



\if0

\subsubsection*{Active Directoryの構成要素と概念}

\begin{itemize}
    \item ドメイン\mbox{}\\
    Acitive Directoryの基本単位を「ドメイン」と呼ぶ。
    Active Directoryで認証を行い、アクセスできる範囲。

\end{itemize}
Active Directoryは単一のサービスではなく、他のサービスを利用することで、機能を実現している。
ユーザ認証に「Kerberos認証」、ディレクトリサービスに「LDAP」を使用している。


\subsubsection*{Active Directory で実現できること}
Active Directoryのメリットとしては、システム管理者はドメインサーバに登録されているすべての
パソコンを一括管理する事ができる。

ユーザはパソコンごとにパスワードとIDを記憶する必要がなくなる。
サーバーにアクセスできる人物を制限できる。


\fi


\chapter{試用するネットワーク構成}
本研究では、L2スイッチ 1台とCentOS8.0のインストールされたサーバ3台を用いて1台をGateway Server、残りの
2台をHost Serverとすることで、
以下図\ref{network_graph}のような隔離ネットワークを構成し、そのネットワーク内でソフトウェアを試用する。
ここで隔離ネットワークとは、
内部のHost Server 2台はGatewayServerを経由しないと外部からは接続できないようになっているネットワークと定義する。
L2スイッチの設定は、
PC7とGateway間、PC8とGateway間のそれぞれの間は接続可能とし、PC7とPC8間は接続不可能となるように設定した。

\begin{figure}[H]
    \centering
    \includegraphics*[width=1.0\textwidth,page=2]{graphs/network_archtecture.pdf}
    \caption{ネットワーク構成図}
    \label{network_graph}
\end{figure}

\subsubsection{使用機器の説明}
ここでは、HostServer、GatewayServerとL2スイッチの使用機器の説明を行う。\par
まず、HostServer 2台とGatewayServer 1台はそれぞれ同じ機器であり、計3台を使用している。
1台の詳細を下表\ref{server_detail}に記す。

\begin{table}[H]
    \centering
    \caption{サーバーの詳細}
    \includegraphics*[width=1.0\textwidth,page=6]{graphs/telework_list.pdf}
    \label{server_detail}
\end{table}


\if0
\begin{itemize}
    \setlength{\parskip}{0.05cm} % 段落間
    \setlength{\itemsep}{0.05cm}
    \item HostServerとGatewayServer\mbox{}
    \begin{itemize}
        \item (メーカー)\mbox{}\\Supermicro
        \item (モデル)\mbox{}\\SYS-5018D-FN4T
        \item (CPU)\mbox{} \\
        %\begin{itemize}
            %\item (model name)
            Intel(R) Xeon(R) CPU D-1541 @ 2.10GHz
            %\item (cpu MHz)2594.895
        %\end{itemize}
    \end{itemize}
    
\end{itemize}
\fi
そして、次にL2スイッチのスペックの詳細を下表にまとめる。


\begin{table}[H]
    \centering
    \caption{L2スイッチの詳細}
    \includegraphics*[width=1.0\textwidth,page=7]{graphs/telework_list.pdf}
    \label{switch_detail}
\end{table}


%そして、今回導入の規模に合わせて、小規模と大規模に分類し表に表す。
\subsubsection{本研究でのアプリケーション規模の定義}
そして、本研究では、試用するソフトウェアを導入規模で小規模と大規模というように分類し、各種機能で細分化して
評価をしていくが、ここで小規模アプリケーションと、大規模アプリケーションを以下のように定義する。


\begin{itemize}
    \setlength{\parskip}{0.05cm} % 段落間
    \setlength{\itemsep}{0.05cm}
    \item 小規模アプリケーション\mbox{}\\
    上図\ref{network_graph}において、「GatewayServer」だけにインストールするもの、「Client」だけにインストールするもの
    と定義する。
    \item 大規模アプリケーション\mbox{}\\
    上図\ref{network_graph}において、「GatewayServer」、「Client」、「HostServer」のそれぞれにインストール
    を行い、各種設定をおこなう必要があるソフトウェアと定義する。


\end{itemize}

次の章で、小規模ソフトウェアの試用を行う。

\chapter{小規模ソフトウェア}
インストール先がゲートウェイサーバーだけのものや、隔離ネットワークにアクセスするクライアントだけの
ものを小規模のアプリケーションと分類して、評価を行う。
まず最初に、評価表を表\ref{shoukibo} に示す。

\begin{table}[H]
    \centering
    \caption{小規模ソフトウェア比較表}
    \includegraphics*[width=1.0\textwidth,page=4]{graphs/telework_list.pdf}
    \label{shoukibo}
\end{table}




\section{sshuttle}\label{sshuttle}
\subsection*{概要}
sshuttle \cite{sshuttle}は、簡易VPNツールである。GitHubだけでなく、専用のホームページも存在した\cite{sshuttle_page}。
リモートアクセスユーザ(Client)のみにインストールすれば
    使用できる。本論文では、図\ref{network_graph}のClientにインストールを行なった。
\subsection*{インストール方法}
今回、クライアントpcにはMacBookを用いたため、homebrewを使ってインストールを行なった。
他のOSの場合のインストール方法もいくつかここに記す。
\begin{itemize}
    \item Ubuntu \mbox{}\\ \$ apt-get install sshuttle
    \item MacOS \mbox{}\\ \$ brew install sshuttle
    \item Centos \mbox{}\\  \$ git clone https://github.com/sshuttle/sshuttle.git\\\$ cd sshuttle \\ \$ sudo ./setup.py install 
\end{itemize}
Clientのみにインストールすればよいため、とても容易に使用することができる。

\subsection*{使用方法}
今回、図\ref{network_graph}でClientと隔離ネットワークとVPNを形成したい時を想定する。\\
"\$ sshuttle -r pc15@15.65.136.94 10.1.1.0/24"のコマンドで、図\ref{sshuttleのVPN構築イメージ図}のようにVPNを形成できた。
一度VPNを形成すると、多段sshをする必要などなく、自由に隔離ホストにアクセスできるようになる。\par
例えば、一度sshuttleでVPNを構築すると、Clientの端末上で\\ \mbox{\$ ssh pc8@10.1.1.8}のコマンドで
隔離サーバであるPC8に直接ssh接続することができる。


\begin{figure}[h]
    \centering
    \includegraphics[width=1.0\textwidth, page=7]{graphs/network_archtecture.pdf}
    \caption{sshuttleのVPN構築イメージ図}
    \label{sshuttleのVPN構築イメージ図}
\end{figure}


\subsection*{評価}
sshuttleを使用するメリットは、「安全性」と「簡易化」が可能になることである。\par
一般的に、図\ref{network_graph}のような隔離ネットワーク内のホストにアクセスするためには、多段sshやTunnelingを行う必要がある。
多段sshというのは今回の場合、ClientがまずGateway Serverにsshを行い、その後Gateway Serverのコマンドラインから、隔離ホストにssh接続するというように
複数回sshを行うことである。この場合、一度sshを切ってしまうと、再び多段sshする必要である上に、各プロセスごとに多段sshをする必要がある。
しかし、sshuttleを使用することで、一度VPNを構築すると自由に隔離ホストにアクセスすることができるため、
とても容易に隔離ホストに接続できる。\par
また、安全性の面では一般的に多段sshを行う場合は、Gateway Serverを経由するため、外部ネットワークと接続されているGateway Serverに
Clientのssh keyを保存しないといけない。しかし、sshuttleでVPNを構築することでGateway Server を経由せずに隔離ホストと接続するため、
Clientと隔離ホスト間で鍵共有するだけでよいため、インターネット等と接続されているGatewayに鍵を保存するより安全である。


%は、sshやsslなどのtcpベースの暗号化ストリームを使用できず、
%独自の暗号化を最初から実装する必要があります。これは非常に複雑でエラーが発生しやすくなります。
%図のようなネットワーク構成図の場合、隔離ホストにアクセスするためには、クライアントからgateway serverにssh接続を行い、その後再び
%隔離ホストへsshを行う、「多段ssh」や、gateway serverから隔離ホストへ「tunneling」を行う必要が発生する。

\section{sshportal}\label{sshportal}

\subsection*{概要}
`sshportal \cite{sshportal} とは、透過的なSSH要塞サーバーにするソフトウェアである。GitHubだけでなく、
開発者のホームページにもソフトウェアについての説明が存在した\cite{sshportalpage}。
Gateway Serverのみにインストールすれば使用することができる。
sshportalを使用することで、管理者はGateway Serverにアクセスし隔離ホストにログインでき、ユーザーを動的に管理することができる。
これによって複数のユーザーを複数のホストに簡単に割り当てられる様になる。
Gateway Sserver のみが両側に関する情報を知っているため、ユーザはホストを知る必要がなく、アクセスする必要がある
すべてのものに自動的に接続される。




\subsection*{インストール方法}
sshportalはDockerを用いることで容易にインストールができる。
また、今回使用したGateway ServerのOSはCentOS8.0であるが、8.0からDockerと互換性のある"Podman"が追加された。
使用法はDockerとほとんど違いはないため、そのコマンドもここに記す。

\begin{itemize}
    \item Docker\mbox{}\\docker pull moul/sshportal
    \item Podman\mbox{}\\podman pull moul/sshportal
\end{itemize}

\subsection*{使用方法}
ここでは「podman」使用時のコマンドを示す。
\subsubsection*{管理者の場合}
\begin{itemize}
    \setlength{\parskip}{0.1cm} % 段落間
    \setlength{\itemsep}{0.1cm} 
    \item バックグラウンドサーバーを開始する \mbox{}\\podman run -p 2222:2222 -d --name=sshportal -v "\$(pwd):\$(pwd)" -w "\$(pwd)" 
    moul/sshportal:v1.10.0
    \item ログを表示させる\mbox{}\\podman logs -f sshportal
    \item 管理者(admin)としてログイン\mbox{}\\ \# ssh localhost -p 2222 -l admin\\その後 config \textgreater に切り替わる。ここで動的にユーザー登録を行う。
    
\end{itemize}
もし、サーバーにアクセスしたいユーザーがいるときの使用法
\begin{itemize}
    \setlength{\parskip}{0.1cm} % 段落間
    \setlength{\itemsep}{0.1cm} 
    \item 最初にadminホストを作成する\mbox{}\\ config\textgreater  host create user1@10.1.1.8
    \item サーバーに鍵を追加する\mbox{} \\ \$ ssh user1@10.1.1.8 "\$(ssh localhost -p 2222 -l admin key setup default)"

\end{itemize}
ユーザの招待
\begin{itemize}
    \setlength{\parskip}{0.1cm} % 段落間
    \setlength{\itemsep}{0.1cm} 
    \item 例: config\textgreater user invite bob@exaple.com\mbox{}\\  これによってユーザーを招待している。このコマンドでは、リモートサーバーにユーザーを作成するのではなく、sshportal.dbというデータベースににアカウントを作成する。
\end{itemize}
\setlength{\parskip}{0.1cm} % 段落間
\setlength{\itemsep}{0.1cm} 
\subsubsection*{ユーザーの場合}
\begin{itemize}
    \setlength{\parskip}{0.1cm} % 段落間
    \setlength{\itemsep}{0.1cm} 
    \item \$ ssh localhost -p 2222 -l 10.1.1.8 \mbox{}\\
    これにより、user1はGatewayからpc8に接続することができる。

\end{itemize}


\subsection*{評価}
sshportalはユーザの管理が簡単に行えるように設計されており、新規ユーザの追加等を管理者が動的に行う
ことができる。インストール先はGateway Server のみで良いが、
前述の「sshuttle」のようにGithubとは別に専用のサイトは用意されておらず、Githubにも
詳細には仕組みやインストール方法が記されていないため、導入や使用までにすこし試行錯誤が必要である。

その他のデメリットとしては、ユーザーの管理を自動ではなく、動的に管理者が行う必要があるため、
ユーザーが大人数になるほど管理が大変になってしまう。


%\section{FreeIPA}

\chapter{大規模ソフトウェア}
インストール先がGateway Serverだけでなく、隔離ネットワーク内のホスト全てにRole別に設定を行う必要のあるものを
大規模アプリケーションと分類して、評価を行う。
まず最初に、評価表を下表\ref{daikibo}に示す。

\begin{table}[H]
    \centering
    \caption{大規模ソフトウェア比較表}
    \includegraphics*[width=1.0\textwidth,page=5]{graphs/telework_list.pdf}
    \label{daikibo}
\end{table}

\section{Teleport}\label{teleport}

\subsection*{概要}
Teleport \cite{teleport}は、リモートアクセスのためのセキュリティGatewayとなっており、
開発者によると、従来のOpenSShの代わりに使用されることを目的としているという。 

GitHubの他にも、製作した会社のホームページにも解説が存在した\cite{teleportpage}。

有料版と無料版があり、今回は無料版を使用した。
以下にTelportの機能の一部を示す。

\begin{itemize}
    \setlength{\parskip}{0.1cm} % 段落間
    \setlength{\itemsep}{0.1cm}
    \item 単一のSSHアクセスGateway
    \item SSH証明書ベースの認証
    \item 二段階認証
    \item SSHの役割ベースのアクセス制御
    \item セッションの記録を行う。
    \item シングルサインオン(SSO)
\end{itemize}

基本的なアーキテクチャの概要を下図に示す。
\begin{figure}[H]
    \centering
    \includegraphics*[width=1.0\textwidth,page=1]{graphs/teleport_archtecture.pdf}
    \caption{Teleportのアーキテクチャ}
    \label{teleport_archtecture}
\end{figure}


\subsubsection*{Teleportの構成要素}
Teleportの仕組みを解説する前に、用語の説明を行う。


    
        
\begin{itemize}
    \setlength{\parskip}{0.1cm} % 段落間
    \setlength{\itemsep}{0.1cm}
    \item ノード\mbox{}\\「サーバ」または「コンピュータ」と同義語。SSH接続できるもの。
    \item ユーザ\mbox{}\\ノード上で一連の操作を実行できる人や、マシン。
    \item クラスター\mbox{}\\連携して動作するノードのグループであり、単一のシステムとみなすことができる。
    \item 認証局(CA)\mbox{}\\公開鍵/秘密鍵のペアの形式でSSL証明書を発行する。
    \item Teleportノード\mbox{}\\テレポートサービスを実行しているノード。許可されたテレポートユーザがアクセスできる。
    \item Teleportユーザ\mbox{}\\テレポートクラスタへアクセスしたいユーザ。ユーザはユーザ名とパスワードを登録する必要がある。
    \item Teleportクラスタ\mbox{}\\1つ以上のノードで構成され、各ノードは同じCAによって署名された証明書を持つ。
    
    \item Teleport認証局\mbox{}\\テレポートは、認証サービスの機能として2つの内部CAを運用する。一つはユーザ証明書の署名を行い、
    もう一つはノード証明書の署名に使用される。

    
\end{itemize}

\subsubsection{Teleportサービス}
Teleportでは、「ノード」、「認証」、「プロキシ」の3つのサービスが連携して機能している。
Teleportノードは、SSHを使用してリモートでアクセスできるサーバーである。
「OpenSSH」、「TeleportCLIクライアント(tsh)」、「Webブラウザー」を使用することでTeleport
ノードにログインすることができる。

Teleport認証局はは、ユーザとノードを認証し、ノードへのユーザーアクセスを承認し、ユーザーとノードに発行
された証明書に署名することで認証局として機能する。\par
TeleportProxyは、ユーザー資格情報を認証サービスに転送し、認証が成功した後、要求されたノードへの接続を
作成し、WebUIを提供する。

\subsection*{Teleportの仕組み}
\subsubsection*{Teleportの隔離サーバへ接続までの流れ}
下図の詳細なアーキテクチャを使って説明を行う。

\begin{enumerate}[1:]
    \setlength{\parskip}{0.1cm} % 段落間
    \setlength{\itemsep}{0.1cm}

    \item クライアント接続を開始する\mbox{}\\クライアントは、CLIインターフェースやWebブラウザーを使用してプロキシ(Gateway)へSSH接続を始める。
    そのとき、クライアントは証明書を提供する。\\
    \begin{figure}[H]
        \centering
        \includegraphics*[width=0.8\textwidth,page=2]{graphs/teleport_archtecture.pdf}
        \caption{クライアント接続開始}
        \label{teleport_connect}
    \end{figure}


    \item クライアント証明書の認証\mbox{}\\
    プロキシは、送信された証明書が以前に認証サーバ(CA)によって署名されているかどうかを確認する。もし署名がされていなかった場合(初回ログイン時)
    や証明書の有効期限が切れているとき、プロキシは接続を拒否し、パスワードと二段階認証でのログインをクライアントに求める。
    二段階認証は、Google Authenticatorなどを用いて行う。HTTPS経由でプロキシに送信される。
    \begin{figure}[H]
        \centering
        \includegraphics*[width=0.8\textwidth,page=3]{graphs/teleport_archtecture.pdf}
        \caption{クライアント証明書の認証}
        \label{teleport_certificate}
    \end{figure}

    
    \item クライアントが接続要求するテレポートノードを調べる。\mbox{}\\
    \begin{figure}[H]
        \centering
        \includegraphics*[width=1.0\textwidth,page=4]{graphs/teleport_archtecture.pdf}
        \caption{ノードの検索}
        \label{search_node}
    \end{figure}

    このステップで、プロキシーはクラスター内の要求されたノードを
    見つけようとする。プロキシがノードのIPアドレスを見つける検索メカニズムは3パターンある。
    \begin{enumerate}[(1)]
        \setlength{\parskip}{0.1cm} % 段落間
        \setlength{\itemsep}{0.1cm}
        \item DNSを使用して、クライアントから要求された名前解決を行う。
        \item 登録されているノードがあるかどうか認証サーバに尋ねる。
        \item 要求された名前と一致するラベルを持つノードを見つけるように認証サーバに要求する。
    \end{enumerate}
    その後、ノードが見つかると、プロキシはクライアントと要求されたノード間の接続を確立する。その後、宛先ノードはセッションの記録を開始し、
    セッション履歴を認証サーバ(CA)に送信して保存する。
    \item ノード証明書の認証\mbox{}\\ノードは接続要求を受信すると、認証サーバを使用してノードの証明書を検証しノードのクラスタ
    メンバーシップを検証する。ノード証明書が有効な場合、ノードは、クラスター内のノードおよびユーザーに関する情報へのアクセスを
    提供する認証サーバーAPIへのアクセスを許可される。
    \begin{figure}[H]
        \centering
        \includegraphics*[width=1.0\textwidth,page=5]{graphs/teleport_archtecture.pdf}
        \caption{ノードの認証}
        \label{search_node}
    \end{figure}

    \item ユーザノードのアクセスを許可する\mbox{}\\ノードは認証サーバーに、接続しているクライアントのOS
    ユーザーのリスト(ユーザーマッピング)を提供するように要求し、クライアントが要求されたOSログインの使用を許可されている
    ことを確認する。
    
    \begin{figure}[H]
        \centering
        \includegraphics*[width=1.0\textwidth,page=6]{graphs/teleport_archtecture.pdf}
        \caption{ノードへのアクセス許可}
        \label{search_node}
    \end{figure}
    

    \item 最後に、クライアントはノードへのSSH接続を作成することを許可され、目的のノードへのSSH接続を行う。
    \begin{figure}[H]
        \centering
        \includegraphics*[width=1.0\textwidth,page=7]{graphs/teleport_archtecture.pdf}
        \caption{SSH接続の完了}
        \label{search_node}
    \end{figure}
    
\end{enumerate}


\subsection*{Teleportの評価}
Teleportは、ProxyServer (Gateway)、AuthServer、遠隔アクセス先端末のそれぞれに対応したソフトウェアをインストールし、設定を
行う必要がある。
Teleportでは、外部からProxyへのアクセスがあった際、接続者の持つSSH証明書をAuthServerが検証し、Authに登録された権限
(管理者、アルバイト、開発者等)に沿って、接続者が要求する隔離サーバへのTunnelを引いている。
Teleportを利用することで、誰に対してどのサーバにどんな権限でアクセスさせたいかを統合管理でき、
シェル内で入力されたコマンドは全て記録されるため、誰がいつ何をしたのかを動画として確認することができる。

%また、Teleportでは、WebUIが提供されており、管理者としてログインすると、Proxyの配下にある隔離サーバへ


%Telelportでは、AuthServerに登録されてる権限
%\subsection*{使用方法}




\begin{comment}
    
    
    \section{Aker}
    
    \subsection*{Akerとは}
    Aker(https://github.com/aker-gateway/Aker)は、独自のLinux sshジャンプ/要塞ホストの構成に役立つセキュリティツールである。
    国境を守ったエジプト神話の上にちなんで名付けられている。
    
    すべてのシステム管理者とサポートスタッフがLinuプロダクションサーバーにアクセスするためのチョークポイントとして機能する。
    
    \subsection*{特徴}
    
\end{comment}



\section{SoftEther VPN}\label{SoftEtherVPN}

\subsubsection*{SoftEther VPNとは}

SoftEther VPN \cite{softethervpn}とは、筑波大学における学術目的の研究プロジェクト「SoftEther プロジェクト」により
運営されている\cite{softethervpnpage}、オープンソースソフトウェアである。Windows,Linux,Mac,FreeBSDおよびSolaris上で動作する。

\subsubsection*{SoftEther VPNの特徴}
SoftEther VPNは、カプセル化およびトンネリングの通信をレイヤ2、のデータリンク層で行なっている。
SoftEther VPNを使用することで、通常のLANカード、スイッチングHUB及びレイヤ3スイッチなどの
ネットワークデバイスをソフトウェアによって仮想的に実現し、それらの間をTCP/IPプロトコルをベースとした、
「SoftEther VPN プロトコル」と呼ばれるトンネルで接続することで、柔軟性の高いVPN構築を実現している。

\subsubsection{SoftEtherVPNの使い方}
SoftEtherVPNは様々な種類のVPNの利用パターンを想定している。
その一部を以下に挙げる。

\begin{itemize}
    \setlength{\parskip}{0.05cm} % 段落間
    \setlength{\itemsep}{0.05cm}
    \item 企業内におけるVPN
    \begin{itemize}
        \item PC間接続VPN
        \item リモートアクセスVPN
        \item 拠点間接続VPN
    \end{itemize}
    \item クラウドにおけるVPN
    \begin{itemize}
        \item ローカルPCをクラウドへ参加させる方法
        \item クラウドVMを企業内LANに参加させる方法
        \item クラウドとLANのブリッジVPN接続
        \item 複数クラウド間のVPNブリッジ接続
    \end{itemize}
    \item モバイルにおけるVPN
    \begin{itemize}
        \item iPhoneおよびAndroid
        \item WindowsやMacのモバイルPC
    \end{itemize}
\end{itemize}


本研究では、「リモートアクセスVPN」用にSoftEtherVPN Server、Clientを各種インストールし、設定を行う。

\subsection{リモートアクセスVPN (SoftEtherVPN)}
SoftEtherVPNには、「VPN Server」、「VPN Client」、「VPN Bridge」の大きく3種類の要素がある。
リモートアクセスVPNの実現には、「VPN Server」と「VPN Client」を用いる。

SoftEtherVPNを使用したリモートアクセスVPNでの、イメージ図を下図\ref{remoteaccess}に記す。
\begin{figure}[H]
    \centering
    \includegraphics*[width=1.0\textwidth,page=1]{graphs/softetherVPN.pdf}
    \caption{SoftEtherVPNのリモートアクセス利用}
    \label{remoteaccess}
\end{figure}



リモートアクセスVPNを構築するためには、仮想的なネットワークセグメントと物理的なEthernetネットワークセグメント
との間を「ローカルブリッジ接続機能」により接続する必要がある。
そうすることにより、VPNを経由して仮想HUBに接続下すべてのリモートコンピュータは、物理的な既存のEthernetセグメントの
一部として扱われるようになる。

\subsubsection*{ローカルブリッジとは}
「ローカルブリッジ接続機能」を使用すると、VPN ServerまたはVPN Bridge内で動作している仮想HUBと、そのサーバーコンピュータ上
に接続されている物理的なLANカードとの間をレイヤ2(データリンク層)で接続し、元々別々のEthernetセグメントとして
動作していた2つのセグメントを一つのセグメントに結合することができる。\par
ローカルブリッジにより、仮想HUBに接続しているコンピュータと物理的なLANに接続しているコンピュータの間で互いに
相手が物理的には別のネットワークに接続されているにも関わらず、論理的には同一のEthernetセグメントに接続されている
事になり、Ethernetのレベルで自由に通信することができるようになる。\par
ローカルブリッジを使用することで、リモートアクセス型VPNおよび拠点間接続型のVPNを簡単に構築することができる。\par
ローカルブリッジを使用する上での注意点として、ローカルブリッジ用に新しい物理的なLANカードを増設する必要がある
ということである。ローカルブリッジ接続先として使用したいLANカードが、そのVPN ServerやVPN Bridgeとして通常の
通信に使用しているLANカードであってはいけないのである。

\subsection*{SoftEtherVPNのインストールとリモートアクセスVPNの構築}
SoftEtherVPNには、「VPN Client」、「VPN Server」のそれぞれ別々の種類のソフトウェアが用意されている。そのため、
それぞれに対するインストール方法と設定をここに記す。
本研究でのネットワーク構成に合わせたリモートアクセス型VPNの図を下図\ref{softether}に表す。

\begin{figure}[H]
    \centering
    \includegraphics*[width=0.9\textwidth,page=2]{graphs/softetherVPN.pdf}
    \caption{リモートアクセスVPN}
    \label{softether}
\end{figure}
\subsubsection*{SoftEther VPN ServerのDHCP設定}
ここでは、リモートアクセス型のSoftEtherVPNの利用にはDHCPサーバが必須であるため、
上図\ref{softether}のVPNServer (Centos)をDHCPを行えるよう設定する。あらかじめ"SELinux"は無効化している。\par
まず、\$ sudo yum -y install dhcp でDHCPをインストールする。その後以下のようにDHCPサーバ設定ファイル(/etc/dhcp/dhcpd.conf)を作成した。
ここでは、10.1.1.0/24でIPアドレスを自動で割り当ててくれるように記述した。
\begin{lstlisting}
#
# DHCP Server Configuration file.
#   see /usr/share/doc/dhcp-server/dhcpd.conf.example
#   see dhcpd.conf(5) man page
subnet 10.1.1.0 netmask 255.255.255.0 {
        range 10.1.1.3 10.1.1.254;
        option routers 10.1.1.1;
        option broadcast-address 10.0.0.255;
        option domain-name-servers 10.1.1.1;
} 

\end{lstlisting}
\begin{itemize}
    \setlength{\parskip}{0.0cm} % 段落間
    \setlength{\itemsep}{0.0cm} 
    \item \$ sudo systemctl start dhcpd \\(DHCPサーバの起動 )
    \item \$ sudo systemctl enable dhcpd \\(DHCPサーバの自動起動設定)
\end{itemize}

その後、クライアント側のDHCPをONにすることで、IPアドレスの自動割当が行える。

\subsubsection*{SoftEther VPN Serverのインストールと設定}
ここでは、まずVPNServer側のインストールと設定をまとめる。\par

まず、https://www.softether-download.com/ja.aspx?product=softether からlinux用のものをダウンロードする。

\begin{enumerate}
    \setlength{\parskip}{0.0cm} % 段落間
    \setlength{\itemsep}{0.0cm} 
    \item \$ tar -xzvf softether-vpnclient-v4.28-9669-beta-2018.09.11-linux-x64-64bit.tar.gz\\ (解凍)
    \item \$ cd vpnclient/
    \item \$ make \\ (makeコマンドでビルドを行う。ライセンスの同意を求められるため、1を入力して同意する。)
    \item \$ cd ../
    \item \$ sudo mv vpnclient/ /usr/local/ \\ (生成されたディレクトリを /usr/local/ に移動する。)
    \item \$ cd /usr/local
    \item \$ sudo chown -R root:root vpnclient
    \item \$ cd vpnclient
    \item \$ sudo chmod 600 *
    \item \$ sudo chmod 700 vpncmd
    \item \$ sudo chmod 700 vpnclient \\(パーミッションを設定する。)
    
\end{enumerate}

サーバー側に、サービスファイル(/etc/systemd/system/vpnserver.service)を定義する。
\begin{lstlisting}
[Unit]
Description=SoftEther VPN Server
After=network.target network-online.target

[Service]
ExecStart=/usr/local/vpnserver/vpnserver start
ExecStop=/usr/local/vpnserver/vpnserver stop
Type=forking
RestartSec=3s

[Install]
WantedBy=multi-user.target

\end{lstlisting}
systemctlコマンドでサービスの開始と有効化を行う。
\begin{enumerate}
    \setlength{\parskip}{0.0cm} % 段落間
    \setlength{\itemsep}{0.0cm} 
    \item \$ sudo systemctl start vpnserver
    \item \$ sudo systemctl enable vpnserver
\end{enumerate}

次にSoftEther VPN Serverの設定を行う。
設定はコマンドラインツール「vpncmd」を使って設定した。ネットワークが繋がっていれば、vpncmdを使ってリモート設定も行える。


\begin{figure}[H]
    \centering
    \includegraphics*[width=1.0\textwidth,page=1]{graphs/serverlogin.png}
    \caption{vpncmd起動画面(Server)}
    \label{softetherconfig}
\end{figure}


vpncmd起動後は、Helpを参照しながら必要に応じて拡張機能の各種設定を行う。
\begin{description}
    \setlength{\parskip}{0.0cm} % 段落間
    \setlength{\itemsep}{0.0cm} 
    \item[VPN Server\textgreater] ServerPasswordSet \\ (管理者パスワード設定)
    \item[VPN Server\textgreater] BridgeCreate \\(ローカルブリッジ接続を作成する)
    \item[VPN Server\textgreater] IPsecEnable \\(L2TP/IPSecサーバーを有効化する)

\end{description}

次に、仮想HUBの設定を行う。本研究では、仮想HUB名を「Subthemehub」として使用するユーザーを「shuto」とした。
vpncmdで仮想HUB設定モードに移行するには、起動時に仮想HUB名を入力するか、Server設定モードで以下のコマンド入力すれば良い。


\begin{description}
    \setlength{\parskip}{0.0cm} % 段落間
    \setlength{\itemsep}{0.0cm} 
    \item[VPN Server\textgreater] hublist\\(仮想HUB一覧出力)
    \item[VPN Server\textgreater] hubCreate subthemehub \\(subthemehubという名前の仮想HUBを作成する)
    \item[VPN Server\textgreater] hub subthemehub \\(subthemehubの設定モードに移行する)
    \item[VPN Server/subthemehub\textgreater] 
\end{description}
仮想HUBの新規作成ができたら、仮想HUB設定モードでSoftether VPNを利用するユーザの登録を行う。
\begin{description}
    \setlength{\parskip}{0.0cm} % 段落間
    \setlength{\itemsep}{0.0cm} 
    \item[VPN Server/subthemehub\textgreater] userlist(ユーザーの一覧出力)
    \item[VPN Server\textgreater] userCreate shuto \\(shutoという名前のユーザーを作成する)
    \item[VPN Server\textgreater] userpasswordset shuto \\(shutoの認証をパスワード認証に切り替える)
\end{description}

\subsubsection*{SoftEther VPN Clientのインストールと設定}
SoftEther VPN Clientのインストール方法は、上記のServerとほぼ同じコマンドであり、serverをclientに置き換えるだけである。
MacOSでVPN Clientを使用するには、tuntapのインストールが必須である。homebrewで簡単にインストールできる。
tuntapは仮想LANカードの作成に必要になる。\par
SoftEther VPN clientの起動は、centosの場合はVPN serverとほぼ同じコマンドだが、
今回使用したMacでは、\$ sudo ./vpnclient start と入力する必要がある。\par
次に、VPN Clientの設定を記す。VPN Serverと同じように「vpncmd」を使ってコマンドラインで設定を行なった。


\begin{figure}[H]
    \centering
    \includegraphics*[width=1.0\textwidth,page=1]{graphs/clientconfig3.png}
    \caption{vpncmd起動画面(Client)}
    \label{clientconfig}
\end{figure}
VPNClintでは、始めに利用する仮想LANカードを作成する必要がある。本研究では「SubthemeNic」という名前のLANカードを
作成した。
\begin{description}
    \setlength{\parskip}{0.0cm} % 段落間
    \setlength{\itemsep}{0.0cm} 
    \item[VPN Client\textgreater] NicCreate SubthemeNic  (新規仮想LANカードの作成)
\end{description}
仮想LANカードの作成が完了したら、次に接続設定を行う。

\if0
\begin{description}
    \setlength{\parskip}{0.0cm} % 段落間
    \setlength{\itemsep}{0.0cm} 
    \item[VPN Client\textgreater] AccountCreate \\新規接続設定の作成

\end{description}
\fi
\begin{figure}[H]
    \centering
    \includegraphics*[width=0.8\textwidth,page=1]{graphs/accountcreate.png}
    \caption{接続設定作成画面}
    \label{accountcreate}
\end{figure}

VPN Serverでの設定でパスワード認証を設定したため、VPN Clientでも同じパスワードを登録する必要がある。
\begin{description}
    \setlength{\parskip}{0.0cm} % 段落間
    \setlength{\itemsep}{0.0cm} 
    \item[VPN Client\textgreater] AccountPasswordSet Subtheme\\ これによりVPNClientの設定が完了した。
\end{description}


\subsubsection*{SoftEtherVPN Serverへの接続}
最後に、VPNClientから、VPNServerに接続する。
\begin{description}
    \setlength{\parskip}{0.0cm} % 段落間
    \setlength{\itemsep}{0.0cm} 
    \item[VPN Client\textgreater] accountconnect subtheme(使用する接続設定を指定)\\ 
    \item[\$] sudo ipconfig set tap0 dhcp \\(ここからvpncmdを抜けて、コマンドラインで行う。このtap0とは接続設定で
    使用したNICと同じMACアドレスを持つものを指定する)
    \item[\$] ifconfig tap0 \\
    \begin{figure}[H]
        \centering
        \includegraphics*[width=0.8\textwidth,page=1]{graphs/dhcpsuccess.png}
        \caption{ifconfig tap0の出力}
        \label{tap0の出力}
        結果、隔離ネットワークと同じセグメントのIPアドレスを取得できた。
    \end{figure}

\end{description}
そして、最後にVPN Client\textgreater AccountStatusGet subtheme で接続状況を確認すると以下\ref{accountstatusget}
の出力が得られた。これを見ると、暗号スイートが"TLS\_AES\_256\_GCM\_SHA384"をしており、プロトコル詳細を見ると
TLSにはOpenSSLを使用して、通信内容の暗号化にchacah20という共通鍵暗号とPoly1305というメッセージ認証方式を使用している
ことがわかる。

\begin{figure}[H]
    \centering
    \includegraphics*[width=1.0\textwidth,page=1]{graphs/accountstatusget.png}
    \caption{接続状況の出力画面}
    \label{accountstatusget}
\end{figure}


\if0
\subsection*{SoftEther VPN Server の仕様}
同時接続可能な最大VPNセッション数
4096個
ユーザ認証:
パスワード認証
RADIUS認証
Active Directory 認証
固有証明書認証
署名済み証明書認証
\fi





\subsection*{SoftEther VPNの評価}
SoftEtherVPNは、GatewayServer、利用するClient、のそれぞれに対応したソフトウェア(VPNServer、VPNClient、VPNBridge)をインストールし設定を行う必要がある。
しかし、Githubにだけでなく専用のサイトが存在しそこにOSごとの詳細な説明が記載されているともに、日本発祥のソフトウェアであることにより
"Qiita"などにも多くの情報が存在するため、とても容易にインストールと設定を行うことができた。
今回は、認証方式をパスワード認証にしたが、その他にもRADIUS認証やサーバ証明書認証なども行える。また、今回
利用した"リモートアクセス型"だけでなく、その他十種類以上の利用形態が想定されており、
とても安全性、柔軟性、拡張性に富んだソフトウェアといえる。

デメリットとしては、SoftEtherVPNServerをインストールするサーバがDHCPサーバであることが必須であることだ。
これにより個人的に利用する場合はGatewayServerを各自DHCPサーバ化する必要がある。サーバ等の知識が必要がなってくる。





\chapter{通信制御に関する考察}

本論文では、前章までにかけて通信制御や遠隔アクセスを容易にするアプリケーションを試用してきた。
この章では、通信接続やその接続の管理を容易にするために必要な機能とそれを実現するための技術を考える。
自分が考える必要な機能と実現する技術を以下に挙げ、それを以下表\ref{spec_and_technic}と表\ref{software}にまとめた。

\begin{enumerate}
    \setlength{\parskip}{0.1cm} % 段落間
    \setlength{\itemsep}{0.1cm} 
    \item 接続段数を減らしたい。
    \item 接続の際のトンネルの管理の容易にしたい。%トンネリング
    \item ログイン時に入力の手間を省きたい。%証明書
    \item ユーザ情報の一括管理が行える。%LDAP,Kerberos
    \item 初期登録時、一括管理に加えて登録の自動化を行える。
    \item GUIが利用できる。
    \item 接続ログやセッションログが取れたら良い
    \item 利用端末と遠隔アクセス端末側(E2E)のみアカウントがあれば利用できる。%VPN
\end{enumerate}

\begin{table}[h]
    \caption{求める機能と技術とソフトウェア}
    \centering
    \includegraphics*[width=0.9\textwidth,page=8]{graphs/telework_list.pdf}
    \label{spec_and_technic}
\end{table}
\begin{table}[h]
    \caption{求める機能とソフトウェアの対応表}
    \centering
    \includegraphics*[width=0.9\textwidth,page=9]{graphs/telework_list.pdf}
    \label{software}
\end{table}

\begin{figure}[h]
    \centering
    \includegraphics*[width=0.9\textwidth,page=20]{graphs/network_archtecture.pdf}
    \caption{VPNイメージ}
    \label{VPN_image}
\end{figure}

\subsubsection*{接続段数を減らしたい}

システム管理者やユーザにとって、外部から隔離ネットワーク内のサーバにアクセスする際に最も煩わしく感じる場面は、
例えばいくつものGatewayを経由して、多段SSHしながら目的のサーバに接続を行う作業であろう。
ホスト名とユーザ名を指定して何度もアカウント情報とコマンドを入力するのは大変面倒なことである。
このような場面で必要となる最も代表的な技術は\ref{aboutVPN}節で説明した「VPN」である。
VPNを設定することで、図\ref{VPN_image}のように利用する端末と接続先の端末の存在するネットワーク間で仮想的なネットワークを引くことができ、
殆どの場合一度の接続コマンドで目的を達成できる。
\par 本論文で試用したソフトウェアで同様のことができるものは、以下の2つである。
\begin{itemize}
    \item sshuttle (\ref{sshuttle}節)
    \item SoftEtherVPN (\ref{SoftEtherVPN}節)
\end{itemize}\par
「sshuttle」は、遠隔アクセスを行う端末にインストールするだけで、sshuttleコマンドと引数に
Gatewayのホスト名かIPアドレスと隔離ネットワークのサブネットを
指定することで
隔離ネットワークに対して簡易的なVPNを形成してくれるソフトウェアである。

「SoftEtherVPN」は、遠隔アクセス端末と接続先端末の両方に対応したコンポーネントを適用することで、
レイヤ2でのVPNを実現するソフトウェアである。SoftEtherを利用することで実際に隔離ネットワーク内のIPアドレスを
割り当ててもらうことができ、実際のEthernetに接続されているように実感することができる。
\par 
「VPN」によって、接続先サーバと仮想的に同じネットワークに所属させることで接続コマンド入力の回数を減らす方法の他に、
\ref{Tunnel}節で説明したように「トンネリング」よって回数を減らすこともできる。
隔離サーバとGateway間、利用端末とGatway間にトンネルを引くことで少ない段数で接続を完了することができる。

\par 本論文で試用したソフトウェアで同様のことができるものは、以下の4つである。
\begin{itemize}
    \item sshuttle (\ref{sshuttle}節)
    \item sshportal (\ref{sshportal}節)
    \item Teleport (\ref{teleport}節)
\end{itemize}
「Teleport」では、認証を行った後隔離サーバとGateway
間でトンネルが引かれている。これらの他に、OpenSSHもトンネリングを行うことができる。


\subsubsection*{トンネルの管理の容易にしたい}
前述したように、接続の段数を減らすために「トンネリング」の技術が用いられる。
段数を減らすことができるが、最初のアクセス時には目的サーバまでのトンネルを引く必要がある。
そのため、ユーザは対応するコマンドを入力することでトンネルを作成し、目的地までそのコマンド何度も入力するのは面倒である。
また、システム管理者はGatewayを経由して引かれたトンネルを管理する必要がある。
使われなくなったトンネルを切断したり、悪意のある使用がされていないか、誰が使用しているかなどを知る必要がある。

今回試用したソフトウェアで「トンネルの管理を容易化」の機能に該当するものはなかった。


\subsubsection*{ログイン時に入力の手間を省略したい}
ユーザや管理者にとって、遠隔アクセスやサーバ利用の際に上記の「接続先への段数を減らす」ことの他に、
「ログイン時の入力の手間」を削減できるのであれば、とても便利なことであろう。
「接続先への段数を減らす」ことができたとしても、他のサーバアクセスや機器利用の際に再びパスワード等の情報を入力するのは
大変面倒である。

このような場面で必要となる技術はいくつかあるが、「Kerberos認証による、シングルサインオン(SSO)」や「SSH公開鍵認証の利用」、
「SSH証明書認証の利用」である。
\ref{KerberosAuthority}節で解説したように、Kerberos認証を用いることでシングルサインオン(SSO)を実現でき、一度のログインのみで
範囲内のサーバや機器に自由にアクセスすることができる。他のテクニックとして、SSHにおいて公開鍵暗号を遠隔
アクセス利用端末で作成し、接続先端末に登録しておけばアカウントを入力する手間を省くことができる。
同様に、SSHで証明書認証を利用することで、SSHコマンドを入力すれば自動的にログインすることができる。
\par 今回試用した全てのソフトウェアで、同様のことができる。

\if0
sshの回数
パスワード認証でなく、
公開鍵を接続先に登録することでコマンド入力の煩わしさが減る。
公開鍵認証を行うようにすると良いかも
これを達成するための技術はVPN、やトンネルを作成する。
\fi

\subsubsection{ユーザの情報の一括管理を行える}
システム管理者にとって面倒な作業だと感じる瞬間の1つは、一括管理されていないサーバに新規ユーザ登録の時であろう。
一括管理されていないために、1台1台にroot権限で入ったのち新規ユーザの登録を行わなければならない。
そして、ユーザからすれば常に同じサーバのところに赴きログインしなければいけなくなる。
なぜなら、異なるサーバだとアカウント情報が登録されてないからである。


今サーバ利用の例を出したが、LANへの接続利用の場合も似たようなことがいえる。
LAN接続のためのアカウント情報が一括管理されていない場合は、 ユーザは学内や社内LANに接続する際に、
アカウントが登録されている機器にしか接続できなくなってしまう。

これらの場面で必要になる技術は、\ref{Authentication}章の中で解説した、「LDAP」や「Kerberos」、「RADIUS」である。
「LDAP」、「Kerberos」を利用することで、サーバのアカウント情報を一括管理することができる。
これにより、管理者はLDAPサーバ、KDCサーバにユーザ情報を登録するだけで、範囲内のサーバにアカウント情報を適用
させることができ、ユーザはどのサーバからでもログインすることができる。
また、ネットワーク接続においては、「RADIUS」を利用することでネットワークアカウント情報を一括管理でき、
ユーザはどの機器からもネットワークに接続することができる。
加えて、「LDAP」と「RADIUS」を連携させることで、サーバのアカウント情報をネットワークアカウント情報と兼ねることができる。

\par 本論文で試用したソフトウェアで同様のことができるものは、以下の4つである。
\begin{itemize}
    \item Active Directory (\ref{Active Directory}節)
    \item sshportal (\ref{sshportal}節)
    \item Teleport (\ref{teleport}節)
    \item SoftEtherVPN (\ref{SoftEtherVPN}節)
\end{itemize}

「sshportal」では、「LDAP」等の技術を利用していないがデータベースにアカウント情報を登録することでユーザのログイン
認証を行っている。開発者によると、今後sshportal v2においてLDAPなどの認証技術を利用できるようにするようである。

「Active Directory」では、ディレクトリーサービスの実現に「LDAP」、ユーザ認証の実現に「Kerberos」の技術を用いている。
「Teleport」、「SoftEtherVPN」では、オプションとして「Active Directory」や「RADIUS」と連携できるようになっている。

\subsubsection*{一括管理における登録の自動化を行える}
前述のアカウント情報の一括管理に加えて、情報の登録作業を自動化できるのならば、管理者にとってとても便利なことであろう。
管理者が何十人、何百人もの情報を登録しなければならない状況であった時、
1人1人手打ち入力で登録していくのはとても面倒な作業である。
\par そういう場合の「技術」というものは見つからなかった。
調べた中にあったものは、シェルスクリプトを用いて一度に行うものや、エクセルを用いてできるだけ入力作業を減らす
テクニックは存在した。

\subsubsection*{GUIが利用できる}
一般的に管理者はCUIで作業をすることが多い。
大規模システムにおいて、CUIでの作業はシステムの一括制御がしやすくなるためである。
しかし、CUIだけでなくGUIも利用することができるのならば、視覚的にログイン状況や、サーバの利用率、負荷を確認できる
ようになり、より管理しやすくなるだろう。
\par 今回試用したソフトウェアで、GUIを実現している2つを以下にあげる。
\begin{itemize}
    \item Teleport (\ref{teleport}節)
    \item SoftEtherVPN (\ref{SoftEtherVPN}節)
\end{itemize}
「Teleport」では、クライアントから接続可能なサーバのリストを表示したり、それらに対してのWebベースのターミナルを開いたり、
記録されたセッションを表示したり再生したりすることができる。


\subsubsection*{接続ログやセッションログの取得したい}
管理者にとって、外部から隔離ネットワークに接続してくるユーザのアクセスログを取得できることは、
セキュリティの向上につなげることができる。

\par 本論文で試用したソフトウェアで同様のことができるものは、以下の3つである。
\begin{itemize}
    \item sshportal (\ref{sshportal}節)
    \item Teleport (\ref{teleport}節)
    \item SoftEtherVPN (\ref{SoftEtherVPN}節)
\end{itemize}


\subsubsection*{エンドツーエンド(E2E)のみのアカウント情報の利用できる}
外部から隔離ネットワーク内のサーバにアクセスする際に、経由する全てのサーバのアカウント情報知る必要があるのは大変
煩わしいことである。遠隔アクセス利用端末と接続先端末の、エンドツーエンドに関する情報だけ知ればよくなれば大変便利である。
このような場面で必要となる技術は、「SSHの段数を減らす」際に用いられた、「VPN」技術である。
VPNを用いることで、利用端末と接続先端末間で仮想ネットワークを構築するため、遠隔アクセスには
2つのアカウント情報のみで十分となる。

\par 本論文で試用したソフトウェアで同様のことができるものは、以下の2つである。
\begin{itemize}
    \item sshuttle (\ref{sshuttle}節)
    \item SoftEtherVPN (\ref{SoftEtherVPN}節)
\end{itemize}\par

\if0
また、その情報の登録自体も容易にできたら良い。
VPNによって、利用端末と遠隔の端末だけアカンウトがあればログインできる。
GatewayServerを中継している、接続ログが取れたら良い。
利用期限によって、パスワード管理など安全性を高めたい。(二週間程度ならそこまでセキュリティ高くなくてもよい)
(1年以上などの場合は高いセキュリティが欲しい。)
台数(手間と時間のバランス)
ネットワークを伸ばすトンネルを作る、トンネル管理、
パスワード、などあまり打つ必要がない方が嬉しい。
\fi


\chapter{おわりに}
本論文では前章にかけて、遠隔アクセスのために使われる技術とそのセキュリティ性向上のために使われている技術を解説した後、
それらの技術を用いているいくつかのオープンソースソフトウェアを試用し、通信制御に関する考察を行った。
本章では全体のまとめを行う。\par
本研究では、ソフトウェアを試用するために3台のサーバと1台のL2スイッチで隔離ネットワークを構成しなければならなかった。
サーバにはそれぞれにOSを入れ、スイッチには適した設定を自分で行い、ソフトウェアを1つ1つインストールしていき、性能を
調査した。
インストールするソフトウェアは「Github」から、できるだけお気に入りやフォークが多いものを選んだ。
研究開始時、お気に入り数が少ないものも選んでいたが、お気に入り数が少ないものは
個人目的のために作られているもの、ソフトウェアの更新がだいぶ昔に止まっているもの、エラーの多いものや、
インストール方法や設定が記されていないものがほとんどであったため、結局それらお気に入り数の少ないものを除外し多くの
人の目に止まっているソフトウェアを選択した。

\if0
Githubで通信制御のためのソフトウェアを検索する中で気になったことは、「ssh」や「VPN」に関するソフトウェアの作成者で
上位にくるものの多くが、設定に関する説明文が中国語のみで、中国人向けに作られたもの多くあった。

\fi

前章の考察において羅列した、求める機能を最も多く満たしていたソフトウェアは「Teleport」と「SoftEtherVPN」であった。
これら2つは大規模ソフトウェアという、Gatewayやクライアントそれぞれに適した設定を行う必要があり、導入難易度が
比較的高いものであったため、それ相応の
高い性能をもっていたと思われる。\par
いくつかソフトウェアを試用する中で個人的にとても便利だと感じたのは、「sshuttle」と「SoftEtherVPN」である。
「sshuttle」は、簡易的なVPNを生成できるツールであるが、
利用するには遠隔アクセスを行う端末側のみにインストールすればよく、Gatewayにsshuttle用の設定を行う必要が
ないため、すぐに「sshuttle」の利便性を感じることができた。\par
「SoftEtherVPN」は、大規模ソフトウェアの中では比較的容易に設定を行うことができる上に、
高い性能を持っているソフトウェアであった。
レイヤ2で機能するものであるため、他のVPNソフトウェアとは違う使用感(実際にIPを割り振られるところなど)を感じることができた。
今回の使用方法の他にもさまざまな使用モデルが想定されており、海外から日本国内の自宅のネットワークを利用できるように
するなど、大変興味深い使用方法が記載されており、将来的に使いこなせるようになりたいと感じた。







%大規模ソフトウェアには、クライアント、Gateway、隔離サーバの3種別の設定を行う必要があったため、それにはとても苦労した。
%「Ansible」や「Vagrant」を自分が使いこなすことができたならば、とても楽に設定を行えるソフトウェアがいくつかあった。



%を使って自分で隔離ネットワーク


\bibliography{link_subtheme,rfc.bib}
\bibliographystyle{junsrt}


\end{document}



